\cvsection{Extracurricular Activity}

\begin{cventries}

  \cventry
    {Maintainer}
    {django-environ}
    {}
    {2021 - PRESENT}
    {
      \begin{cvitems}
        \item {\textbf{Languages:} Python}
        \item {\textbf{Description:} A Python package that allows you to use Twelve-factor methodology to configure your Django application with environment variables.}
        \item {\textbf{URL:} https://github.com/joke2k/django-environ}
      \end{cvitems}
    }

  \cventry
    {Creator, Maintainer}
    {python-airslate}
    {}
    {2021 - PRESENT}
    {
      \begin{cvitems}
        \item {\textbf{Languages:} Python}
        \item {\textbf{Description:} A Python client library has been designed to quickly and easily implement the airSlate REST API.}
        \item {\textbf{URL:} https://github.com/airslate-oss/python-airslate}
      \end{cvitems}
    }

  \cventry
    {Creator, Maintainer}
    {gstore}
    {}
    {2021 - PRESENT}
    {
      \begin{cvitems}
        \item {\textbf{Languages:} Python}
        \item {\textbf{Description:} A simple tool to synchronize GitHub repositories of your organizations.}
        \item {\textbf{URL:} https://github.com/sergeyklay/gstore}
      \end{cvitems}
    }

  \cventry
    {Creator, Maintainer}
    {esup}
    {}
    {2020 - PRESENT}
    {
      \begin{cvitems}
        \item {\textbf{Languages:} ELisp}
        \item {\textbf{Description:} A GNU Emacs Start Up Profiler.}
        \item {\textbf{URL:} https://github.com/jschaf/esup}
      \end{cvitems}
    }

  \cventry
    {Maintainer}
    {jaeger-client-php}
    {}
    {2019 - PRESENT}
    {
      \begin{cvitems}
        \item {\textbf{Languages:} Apache Thrift, PHP}
        \item {\textbf{Description:} A client-side library that can be used to instrument PHP apps for distributed trace collection, and to send those traces to Jaeger.}
        \item {\textbf{URL:} https://github.com/jonahgeorge/jaeger-client-php}
      \end{cvitems}
    }

  \cventry
    {Creator, Maintainer}
    {bnf-mode}
    {}
    {2019 - PRESENT}
    {
      \begin{cvitems}
        \item {\textbf{Languages:} ELisp}
        \item {\textbf{Description:} A GNU Emacs major mode for editing BNF grammars.}
        \item {\textbf{URL:} https://github.com/sergeyklay/bnf-mode}
      \end{cvitems}
    }

  \cventry
    {Maintainer}
    {idea-plugin}
    {}
    {2019 - 2020}
    {
      \begin{cvitems}
        \item {\textbf{Languages:} Java, Kotlin}
        \item {\textbf{Description:} IntelliJ plugin for editing Zephir code. Provides syntax definition, autocompletion and syntax check support.}
        \item {\textbf{URL:} https://github.com/zephir-lang/idea-plugin}
      \end{cvitems}
    }

  \cventry
    {Core Team Member}
    {Codeception}
    {}
    {2019 - 2020}
    {
      \begin{cvitems}
        \item {\textbf{Languages:} PHP}
        \item {\textbf{Description:} A full-stack testing PHP framework.}
        \item {\textbf{URL:} https://github.com/Codeception/Codeception}
      \end{cvitems}
    }

  \cventry
    {Creator, Maintainer}
    {Qless PHP}
    {}
    {2019 - 2020}
    {
      \begin{cvitems}
        \item {\textbf{Languages:} Lua, PHP}
        \item {\textbf{Description:} PHP Bindings for Qless.}
        \item {\textbf{URL:} https://github.com/pdffiller/qless-php}
      \end{cvitems}
    }

  \cventry
    {Maintainer}
    {zephir}
    {}
    {2016 - 2020}
    {
      \begin{cvitems}
        \item {\textbf{Languages:} C, Zephir, PHP}
        \item {\textbf{Description:} A high level programming language that eases the creation and maintainability of extensions for PHP.}
        \item {\textbf{URL:} https://github.com/zephir-lang/zephir}
      \end{cvitems}
    }

  \cventry
    {Core Team Member}
    {cphalcon}
    {}
    {2015 - 2020}
    {
      \begin{cvitems}
        \item {\textbf{Languages:} C, Zephir, PHP}
        \item {\textbf{Description:} An open source web framework delivered as a C extension for the PHP language providing high performance and lower resource consumption.}
        \item {\textbf{URL:} https://github.com/phalcon/cphalcon}
      \end{cvitems}
    }

\end{cventries}
